% !TeX root = ../hw4.tex
\section{Generating a Simple Fractal Tree}

\subsection{Statement}
\todoinline[caption=Give Problem Statement]{Give problem statement including specification of scaling and intent to play with different angles.}

\subsection{Method}

\subsubsection{Lindenmayer Systems}
\todoinline[caption=Define an L-System]{Define an L-System, production rules, and specify the meaning of the different commands.}

\subsubsection{The Lindenmayer System for the Given Problem}
\todoinline[caption=Give l-system for given problem]{Give the L-System for given problem and discuss how it might be improved or modified.}

\subsubsection{Extending the Lindenmayer Systems to 3D}
\todoinline[caption=Defined new 3D commands]{
    Show how to add more commands to extend the L-Systems into 3D.

    Mention that we implemented this as the book suggested --- by creating a 3D turtle that can yaw, pitch, and roll.

    Give minimal examples of each command.
}

\subsubsection{Further Extensions of the Lindenmayer Systems}
\todoinline[caption=Further extending the L-System]{Discuss how to add color, radii proportionality, and randomness}
\todoinline[caption=Add randomness]{
    Implement randomness.
    I can think of two ways to do so:
    First, add an \texttt{r} command that perturbs the immediately following command.
    This would introduce randomness, but at regular intervals.

    Second, add a randomness parameter to the constructor, so that if it's set, apply perturbations at random.
}

\subsection{Implementation}
\todoinline{
    McGough has seemed to like my (brief) discussion of my previous implementations.
    We should not give the whole thing, just the interesting bits and pieces.
}

\subsubsection{Existing Implementations}
\todoinline[caption=Existing impelemtations]{
    Mention all of the existing implementations we found of L-Systems, both 2D and 3D.
    Give links to the easiest to use implementations.

    Show how to use \LaTeX{} to generate them, because he'd like that, and might even find it useful.
}

\subsubsection{Applying the Production Rules}
\todoinline[caption=Implementation of the production rules]{Show how to apply the production rules.}

\subsubsection{A 3D Turtle}
\todoinline[caption=Implementation of the 3D turtle]{
    Show how the 3D turtle was implemented.
    Be very clear how painful quaternions were and the difficulty of performing the transformations in the local reference frame.

    Cite the Lindenmaker Blender addon \url{https://github.com/lemurni/lpy-lsystems-blender-addon} and thesis, but mention that the only thing we ripped off was the Turtle.
}

\subsubsection{Computing the Vertices}
\todoinline[caption=Implementation of the vertex computation]{
    Discuss the \mintinline{python}{Graphics.compute(lstring)} method and its implementation.
    Be sure to include the data structure it returns and how the material is set.

    Also include the proportional radii, and randomness (once it's implemented).
}

\subsubsection{Adding Cylinders to the 3D Scene}
\todoinline[caption=Implementation of adding objects to the scene]{
    Show how to add a single cylinder to the scene, and mention that it slowed down Blender substantially.

    Show how to fix this by duplicating cylinders, but mention that there was still a substantial slowdown for large fractals.
}

\subsubsection{Parallel Scene Creation}
\todoinline[caption=Implementation of batch processing]{
    Discuss the general idea of the batch mode, but there's no need to give actual code --- other than how to join two files.
}

\subsubsection{Usage}
\todoinline[caption=Script usage]{
    Discuss how to use each of the scripts we created, spending the most time on \texttt{blender.py} and \texttt{batch.sh}.

    Make sure to point out that the \mintinline{python}{bpy} and \mintinline{python}{mathutils} libraries are provided by Blender, and that to use them we need to run our scripts in funky ways.
}

\subsection{Results}

\subsubsection{The Lindenmayer System for the Given Problem}
\todoinline[caption=results for given problem]{
    Show our results for the given problem.

    Show the production rules, and generated graphics.
    Play with the scale and the yaw angle.
}

\subsubsection{Expanding the Lindenmayer Systems to 3D}
\todoinline[caption=3d results]{
    Discuss how to create more interesting 3D fractals, including Koch curves, the dragon curve, and 3D trees/bushes.

    Show the results from \texttt{data/a.json} with the book's parameters.
}

\subsection{Conclusion}
