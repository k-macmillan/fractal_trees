% !TeX root = ../hw4.tex
\section{Reproducing the Gray-Scott Model Results}

\subsection{Statement}
Reproduce patterns $\alpha$, $\lambda$, $\mu$, and $\theta$ in the Gray-Scott Model.

\todoinline{Find a high quality image with the different patterns.}

\subsection{Method}

The Gray-Scott Model models reaction diffusion of two chemicals $U$ and $V$ in an unstirred planar chemical reactor.
Further, we assume the reaction $U \to V$ is autocatalyzed by the presence of $V$.
More specifically, the Gray-Scott model assumes the cubic autocatalysis
\begin{equation}
    U + 2V \to 3V
\end{equation}
with reaction rate $kuv^2$.
This reaction can be modeled according to the following system of differential equations.
\begin{equation}
    \begin{cases}
        \displaystyle{\frac{\partial u}{\partial t} = r_u \nabla^2 u - uv^2 + f(1 - u)} \\[20pt]
        \displaystyle{\frac{\partial v}{\partial t} = r_v \nabla^2 v + uv^2 - (f + k)v}
    \end{cases}\label{prob5:eqn:gray-scott}
\end{equation}
where $u$ and $v$ are the concentrations, and $r_u$ and $r_v$ are the diffusion rates, of $U$ and $V$ respectively.
The chemical $U$ is added to the reactor at the feed rate $f$ which is scaled by the concentration of $U$.
Simultaneously, $U$ and $V$ are drained from the reactor at the kill rate $k$, which is scaled by $f$ and the concentration of $V$.

The system \autoref{prob5:eqn:gray-scott} can be discretized as\footnote{These are legitimately the most painful \LaTeX{} equations I've typeset in the last decade.}
\begin{equation}
    \begin{cases}
        \displaystyle{\frac{u(x, y, t + \Delta t) - u(x, y, t)}{\Delta t}} = r_u \bigg( & \displaystyle{\frac{u(x + \Delta x, y, t) - 2u(x, y, t) + u(x - \Delta x, y, t)}{\Delta x^2} +}         \\
                                                                                        & \displaystyle{\frac{u(x, y + \Delta y, t) - 2u(x, y, t) + u(x, y - \Delta y, t)}{\Delta y^2}\bigg)u - } \\
                                                                                        & \displaystyle{uv^2 + f(1 - u)}                                                                          \\[20pt]
        \displaystyle{\frac{v(x, y, t + \Delta t) - v(x, y, t)}{\Delta t}} = r_v \bigg( & \displaystyle{\frac{v(x + \Delta x, y, t) - 2v(x, y, t) + v(x - \Delta x, y, t)}{\Delta x^2} +}         \\
                                                                                        & \displaystyle{\frac{v(x, y + \Delta y, t) - 2v(x, y, t) + v(x, y - \Delta y, t)}{\Delta y^2}\bigg)v - } \\
                                                                                        & \displaystyle{uv^2 + (f + k)v}
    \end{cases}\label{prob5:eqn:discretized-dogshit}
\end{equation}

and ``simplified'' into
\begin{equation}
    \begin{cases}
        u(x, y, t + \Delta t) = u(x, y, t) + \Delta t \Bigg( r_u \bigg( & \displaystyle{\frac{u(x + \Delta x, y, t) - 2u(x, y, t) + u(x - \Delta x, y, t)}{\Delta x^2} +}         \\
                                                                        & \displaystyle{\frac{u(x, y + \Delta y, t) - 2u(x, y, t) + u(x, y - \Delta y, t)}{\Delta y^2}\bigg)u - } \\
                                                                        & \displaystyle{uv^2 + f(1 - u) \Bigg)}                                                                   \\[20pt]
        v(x, y, t + \Delta t) = v(x, y, t) + \Delta t \Bigg( r_v \bigg( & \displaystyle{\frac{v(x + \Delta x, y, t) - 2v(x, y, t) + v(x - \Delta x, y, t)}{\Delta x^2} +}         \\
                                                                        & \displaystyle{\frac{v(x, y + \Delta y, t) - 2v(x, y, t) + v(x, y - \Delta y, t)}{\Delta y^2}\bigg)v - } \\
                                                                        & \displaystyle{uv^2 + (f + k)v \Bigg)}
    \end{cases}\label{prob5:eqn:discretized-dogshit}
\end{equation}

\todoinline{
    Nuke the above discretization (but still mention it) in favor of the discrete Laplacian with periodic boundary conditions as discussed in \textit{Computational Methods for Inverse Problems}.

    Show the difference between the original Dirichlet boundaries, and the new periodic ones.

    If time allows, run McGough's code to figure out any speed difference.

    Refer to \url{https://en.wikipedia.org/wiki/Discrete\_Laplace\_operator\#Finite\_differences}
}

\subsection{Implementation}
\subsection{Results}
\subsection{Conclusion}
