% !TeX root = ../hw4.tex
\section{Generating a Fractal Landscape}

\subsection{Statement}
Implement the random midpoint displacement algorithm in two dimensions to generate some fractal landscapes.
Examine the influence of $H$ on the generated landscapes.

\subsection{Method}
\autoref{prob3:alg:rmd1d} gives the random midpoint displacement algorithm for generating a random 1-dimensional vector.
Note that the dimensions of this vector are determined by the number of recursive subdivisions of the domain.

\begin{algorithm}[H]
    % \begin{noindent}
    \begin{algorithmic}
        \Function{RandomMidpoint1D}{$nrc$, $\sigma$, $seed$}
            \State{Seed the RNG with $seed$}
            \State{$N \gets 2^{nrc}$}\IComment{$N+1$ is the length of the generated vector}
            \State{$x_0 \gets 0$}\IComment{Initialize the height of both ends of the vector}
            \State{$x_N \gets \sigma \cdot \mathrm{randn()}$}
            \For{$i \in \{1, \dots, nrc\}$}
                \State{$\vec \Delta_i \gets \sigma \cdot 0.5^{(i + 1) / 2}$}\IComment{Compute the variances for each point}
            \EndFor{}
            \State\Call{Recurse}{$\vec x$, $0$, $N$, $1$, $nrc$}
            \State\Return{$\vec x$}
        \EndFunction{}
        \Function{Recurse}{$\vec x$, $t_0$, $t_2$, $t$, $nrc$}
            \State{$t_1 \gets (t_0 + t_2) / 2$}
            \State{$x_{t_1} \gets \frac{1}{2}\cdot(x_{t_0} + x_{t_2}) + \Delta_t \cdot \mathrm{randn()}$}
            \If{$t < nrc$}\IComment{Recursively fill the rest of the vector $\vec x$}
                \State\Call{Recurse}{$\vec x$, $t_0$, $t_1$, $t+1$, $nrc$}
                \State\Call{Recurse}{$\vec x$, $t_1$, $t_2$, $t+1$, $nrc$}
            \EndIf{}
        \EndFunction{}
    \end{algorithmic}
    % \end{noindent}
    \caption{The random midpoint displacement algorithm in one dimension}\label{prob3:alg:rmd1d}
\end{algorithm}

This algorithm effectively generates a random heightmap above a regularly spaced segment of the real number line.
The extension of the algorithm to two dimensions would generate a heighmap above a square 2D grid --- effectively generating a landscape heightmap.

To do so, we must modify \autoref{prob3:alg:rmd1d} to generate a square matrix of random values, and recursively quadsect this matrix until we have reached the recursion limit $nrc$.
This is shown in \autoref{prob3:alg:rmd2d}.

\begin{algorithm}[H]
    % \begin{noindent}
    \begin{algorithmic}
        \Function{RandomMidpoint2D}{$nrc$, $\sigma$, $seed$}
            \State{Seed the RNG with $seed$}
            \State{$N \gets 2^{nrc}$}\IComment{$N+1$ is the length of one side of the square matrix}
            \State{$X_{0,0}, X_{0,N}, X_{N,0}, X_{N,N} \gets \sigma \cdot \mathrm{randn()}$}\IComment{Initialize each of the matrix corners}
            \For{$i \in \{1, \dots, nrc\}$}
                \State{$\Delta_i \gets \sigma \cdot \frac{1}{2}^{(i + 1) / 2}$}
            \EndFor{}
            \State\Call{Recurse2D}{$X$, $(0, 0)$, $(N, 0)$, $(N, N)$, $(0, N)$, $1$, $nrc$}\IComment{Corners in CW order}
        \EndFunction{}
        \Function{Recurse2D}{$X$, $(x_0, y_0)$, $(x_2, y_0)$, $(x_2, y_2)$, $(x_0, y_2)$, $t$, $nrc$}
            \State{$x_1 \gets (x_0 + x_2) / 2$}
            \State{$y_1 \gets (y_0 + y_2) / 2$}
            \State{$X_{x_0, y_1} \gets \frac{1}{2}\big(X_{x_0, y_0} + X_{x_0, y_2}\big) + \Delta_t \cdot \mathrm{randn()}$}
            \State{$X_{x_1, y_0} \gets \frac{1}{2}\big(X_{x_0, y_0} + X_{x_2, y_0}\big) + \Delta_t \cdot \mathrm{randn()}$}
            \State{$X_{x_2, y_1} \gets \frac{1}{2}\big(X_{x_2, y_0} + X_{x_2, y_2}\big) + \Delta_t \cdot \mathrm{randn()}$}
            \State{$X_{x_1, y_2} \gets \frac{1}{2}\big(X_{x_0, y_2} + X_{x_2, y_2}\big) + \Delta_t \cdot \mathrm{randn()}$}
            \State{$X_{x_1, y_1} \gets \frac{1}{4}\big(X_{x_0, y_1} + X_{x_1, y_0} + X_{x_2, y_1} + X_{x_1, y_2}\big) + \Delta_t \cdot \mathrm{randn()}$}
            \If{ $t < nrc$}
                \State\Call{Recurse2D}{$X$, $(x_0, y_0)$, $(x_1, y_0)$, $(x_1, y_1)$, $(x_0, y_1)$, $t+1$, $nrc$}
                \State\Call{Recurse2D}{$X$, $(x_1, y_0)$, $(x_2, y_0)$, $(x_2, y_1)$, $(x_1, y_1)$, $t+1$, $nrc$}
                \State\Call{Recurse2D}{$X$, $(x_1, y_1)$, $(x_2, y_1)$, $(x_2, y_2)$, $(x_1, y_2)$, $t+1$, $nrc$}
                \State\Call{Recurse2D}{$X$, $(x_0, y_1)$, $(x_1, y_1)$, $(x_1, y_2)$, $(x_0, y_2)$, $t+1$, $nrc$}
            \EndIf{}
        \EndFunction{}
    \end{algorithmic}
    % \end{noindent}
    \caption{The 2D extension of \autoref{prob3:alg:rmd1d}}\label{prob3:alg:rmd2d}
\end{algorithm}

Modifying both algorithms to work with fractional Brownian motion involves changing the $\Delta_t$ update step for each recursive level from
\begin{equation}
    \Delta_t = \sigma \cdot \frac{1}{2}^{(t + 1) / 2}\label{prob3:eqn:rmd-Delta}
\end{equation}
to
\begin{equation}
    \Delta_t = \sqrt{\frac{\sigma^2}{2^{2tH}} \big(1 - 2 ^ {2 H - 2}\big)}\label{prob3:eqn:fBm-Delta}
\end{equation}

Then the parameter $H$, called the Hurst exponent, describes the roughness of the fractal.
The book indicates lower values of $H$ should \textit{increase} the fractal roughness, while higher values will decrease the roughness.

Also note that the fractal's dimension can be recovered from $H$
\begin{equation}
    d = 2 - H\label{prob3:eqn:fractal-dimension-1d}
\end{equation}
for the 1D random midpoint algorithm, and
\begin{equation}
    d = 3 - H\label{prob3:eqn:fractal-dimension-2d}
\end{equation}
for the 2D version.
Thus smaller $H$ values \textit{increase} the fractal dimension, while larger values decrease the fractal dimension.

\subsection{Implementation}
After the modification of \autoref{prob3:alg:rmd1d} and \autoref{prob3:alg:rmd2d} to include the Hurst exponent, the implementations are straightforward.

The implementation of \autoref{prob3:alg:rmd1d} is given below.

\begin{minted}{python}
    def rand_displacement_1d(recursions, scale, seed, hurst=0.5):
        """Generate a 1D heightmap using the random midpoint displacement algorithm."""
        np.random.seed(seed)
        N = 2 ** recursions
        x = np.zeros(N + 1)
        x[0], x[N] = scale * np.random.randn(2)
        # Calculate the diminishing variance.
        var = [
            np.sqrt((scale ** 2 / (2 ** (2 * l * hurst))) * (1 - 2 ** (2 * hurst - 2)))
            for l in range(recursions)
        ]

        def _recurse(t0, t2, level):
            """Recursively subdivide and perturb the midpoint with diminishing variance."""
            # NOTE: t1 will always be an integer because N is a power of 2.
            t1 = (t0 + t2) // 2
            x[t1] = 0.5 * (x[t0] + x[t2]) + var[level - 1] * np.random.randn()

            if level < recursions:
                _recurse(t0, t1, level + 1)
                _recurse(t1, t2, level + 1)

        _recurse(0, N, 1)
        return x
\end{minted}

The implementation of \autoref{prob3:alg:rmd2d} is given below.

\begin{minted}{python}
def rand_displacement_2d(recursions, scale, seed, hurst=0.5):
    """Generate a 2D heightmap using the random midpoint displacement algorithm."""
    np.random.seed(seed)
    N = 2 ** recursions
    X = np.zeros((N + 1, N + 1))
    X[0, 0], X[0, -1], X[-1, 0], X[-1, -1] = scale * np.random.randn(4)
    # Calculate the diminishing variance.
    var = [
        np.sqrt((scale ** 2 / (2 ** (2 * l * hurst))) * (1 - 2 ** (2 * hurst - 2)))
        for l in range(recursions)
    ]

    def _recurse(x0, y0, x2, y2, level):
        """Recursively quadsect and perturb the five midpoints with diminishing variance."""
        x1 = (x0 + x2) // 2
        y1 = (y0 + y2) // 2
        var = np.sqrt((scale ** 2 / (2 ** (2 * level * hurst))) * (1 - 2 ** (2 * hurst - 2)))

        X[x0, y1] = 0.5 * (X[x0, y0] + X[x0, y2]) + var[level - 1] * np.random.randn()
        X[x1, y0] = 0.5 * (X[x0, y0] + X[x2, y0]) + var[level - 1] * np.random.randn()
        X[x2, y1] = 0.5 * (X[x2, y0] + X[x2, y2]) + var[level - 1] * np.random.randn()
        X[x1, y2] = 0.5 * (X[x0, y2] + X[x2, y2]) + var[level - 1] * np.random.randn()

        X[x1, y1] = (
            0.25 * (X[x0, y1] + X[x1, y0] + X[x2, y1] + X[x1, y2])
            + var[level - 1] * np.random.randn()
        )

        if level < recursions:
            _recurse(x0, y0, x1, y1, level + 1)
            _recurse(x1, y0, x2, y1, level + 1)
            _recurse(x1, y1, x2, y2, level + 1)
            _recurse(x0, y1, x1, y2, level + 1)

    _recurse(0, 0, N, N, 1)
    return X
\end{minted}

Notice that in both of these implementations the calculation of $\Delta_t$ is done \textit{outside} the recursive function call.
This is to avoid a massive amount of repeated computation that scales exponentially with the recursive depth.
Also notice that an iterative implementation would be possible, but messy.

Further notice that the \mintinline{python}{seed} parameter is \textit{required}.
Even though an unseeded generation can (and does) produce reasonable results, it is not possible to reproduce them.
Reducibility is important for two reasons.

First, there's the standard ``experiments should be reproducible'' argument, which is reason enough on its own.
Second, if a random landscape appeals to the authors in their experiments, they might want to produce different variations of that landscape --- possibly with different values for the Hurst roughness exponent, or possibly different sea level values.

In either case, we produce a unseeded random number, then seed the random number generator with the generated value.
We do this because it is not possible to retrieve the current seed of the Numpy random number generator.

Since the purpose here is to generate a realistic looking landscape, it is conceivable that a user might have a landscape in mind and might wish to kickstart the random displacement with an initial shape.
We did not allow for this, but custom initializations for the four grid corners would be trivial.

If the user has a rough landscape shape in mind, they might wish to initialize more points than just the corners.
After all, while a random seed can generate any conceivable landscape, finding the right random seed for a particular landscape is nontrivial.\footnote{But left as an exercise for the motivated reader.}
Thus is might be important to find a way to initialize more points than just the corners.
However, doing so involves more book-keeping than is pleasant.

However, a user might be able to pick corner points on a regularly spaced grid, and run \autoref{prob3:alg:rmd2d} on each subdivision before stitching together the final results.
This is promising, but the process of ``stitching'' is less easy than it might at first seem.

Doing such a stitching in the right manner would involve finding a way to match the first and second derivatives in both the $x$ and $y$ directions.
Perhaps a kind of ``blur'' operation could produce believable results.

In either case, we chose to stick with the algorithm as given and leave improvements for the landscape designers to figure out.

\subsection{Results}
We chose to implement both the 1D and 2D versions of the Random Midpoint Displacement algorithm, because the 1D version is more obvious, and at the time the details of incorporation the Hurst exponent $H$ were unclear.

\subsubsection{One Dimension}
\autoref{prob3:fig:1d} shows the results of \autoref{prob3:alg:rmd1d} without the Hurst exponent incorporated.
\begin{figure}[H]
    \centering
    \includegraphics[width=0.8\textwidth]{figures/landscapes/1d.eps}
    \caption{An example of \autoref{prob3:alg:rmd1d} (seed 37)}\label{prob3:fig:1d}
\end{figure}

We also wanted to examine the impact of the different tunable parameters on an easier to interpret plot.
\autoref{prob3:fig:1d-num-recursions} shows the impact of the number of recursions used to generate the heightmap with the same seed.
\begin{figure}[H]
    \centering
    \includegraphics[width=0.8\textwidth]{figures/landscapes/1d-num-recursions.eps}
    \caption{The impact of the number of recursions (seed 420)}\label{prob3:fig:1d-num-recursions}
\end{figure}
Notice that, while the level of fidelity increases, the recursive level alone does not produce a realistic heightmap.

We implemented also played with the $\sigma$ parameter in \autoref{prob3:alg:rmd1d}. When the different heightmaps are shown on the same plot, the differences are apparent.
However, when shown on different plots, each of the four heightmaps are indistinguishable for each other.
This is expected, and is precisely the impact that a \textit{scale} should have.

\begin{figure}[H]
    \centering
    \includegraphics[width=0.8\textwidth]{figures/landscapes/1d-scale.eps}
    \caption{The impact of the $\sigma$ parameter (seed 420)}\label{prob3:fig:1d-scale}
\end{figure}

Lastly, we experimented with the Hurst exponent $H$.
\autoref{prob3:fig:1d-hurst-exponent} shows exactly what we would expect --- smaller values produce exceedingly rough landscapes, while large values produce more believable heightmaps.

\begin{figure}[H]
    \centering
    \includegraphics[width=0.8\textwidth]{figures/landscapes/1d-hurst-exponent.eps}
    \caption{The impact of the $H$ roughness parameter (seed 420)}\label{prob3:fig:1d-hurst-exponent}
\end{figure}

\subsubsection{Two Dimensions}
Having a good feel for the scale and recursive level parameters, all we experiment with in two dimensions is the Hurst roughness exponent.
Values of $H$ ranging from $0.6$ to $0.9$ are shown in \autoref{prob3:fig:2d-hurst}.

\begin{figure}[H]
    \centering
    \begin{subfigure}[t]{0.55\textwidth}
        \centering
        \includegraphics[width=\textwidth]{figures/landscapes/2d-hurst-9.eps}
        \caption{$H = 0.9$}
    \end{subfigure}%
    \begin{subfigure}[t]{0.55\textwidth}
        \centering
        \includegraphics[width=\textwidth]{figures/landscapes/2d-hurst-8.eps}
        \caption{$H = 0.8$}
    \end{subfigure}%

    \begin{subfigure}[b]{0.55\textwidth}
        \centering
        \includegraphics[width=\textwidth]{figures/landscapes/2d-hurst-7.eps}
        \caption{$H = 0.7$}
    \end{subfigure}%
    \begin{subfigure}[b]{0.55\textwidth}
        \centering
        \includegraphics[width=\textwidth]{figures/landscapes/2d-hurst-6.eps}
        \caption{$H = 0.6$}
    \end{subfigure}%
    \caption{The impact of the $H$ roughness parameter (seed 3188393957)}\label{prob3:fig:2d-hurst}
\end{figure}

Interestingly, using $H=1$ produces a smooth curve, which in hindsight makes complete sense.

\begin{figure}[H]
    \centering
    \includegraphics[width=0.8\textwidth]{figures/landscapes/2d-hurst-1.eps}
    \caption{$H = 1$}\label{prob3:fig:2d-hurst-1}
\end{figure}

We also played with an optional sealevel.
This might be a mind trick, but we believe the landscapes look more believable with an accompanying body of water.

\begin{figure}[H]
    \centering
    \begin{subfigure}[t]{0.9\textwidth}
        \centering
        \includegraphics[width=\textwidth]{figures/landscapes/2d-sealevel-1.eps}
        \caption{Seed 68230869 with a sealevel of -1}
    \end{subfigure}%

    \begin{subfigure}[b]{0.9\textwidth}
        \centering
        \includegraphics[width=\textwidth]{figures/landscapes/2d-sealevel-2.eps}
        \caption{Seed 4175779808 with a sealevel of -0.7}
    \end{subfigure}%
    \caption{Landscapes with different sealevels}\label{prob3:fig:2d-sealevel}
\end{figure}
