% !TeX root = ../hw4.tex
\section{Generating a Fractal Landscape}

\subsection{Statement}
Implement the random midpoint displacement algorithm in two dimensions to generate some fractal landscapes.
Examine the influence of $H$ on the generated landscapes.

\todoinline[caption=Maybe do this in Blender too]{
    Maybe do this in Blender too, because we can create objects from meshes, which is what the heightmap would give us.

    The only trick would be to figure out how to color \textit{one} mesh with a height-based gradient.
}

\subsection{Method}
Algorithm~\ref{prob3:alg:rmd1d}\todo{For some reason the autorefs for algorithms are broken now?}{} gives the random midpoint displacement algorithm for generating a random 1-dimensional vector.
Note that the dimensions of this vector are determined by the number of recursive subdivisions of the domain.

\begin{algorithm}
    % \begin{noindent}
    \begin{algorithmic}
        \Function{RandomMidpoint1D}{$nrc$, $\sigma$, $seed$}
            \State{Seed the RNG with $seed$}
            \State{$N \gets 2^{nrc}$}\IComment{$N+1$ is the length of the generated vector}
            \State{$x_0 \gets 0$}\IComment{Initialize the height of both ends of the vector}
            \State{$x_N \gets \sigma \cdot \mathrm{rand()}$}
            \For{$i \in \{1, \dots, nrc\}$}
                \State{$\vec \Delta_i \gets \sigma \cdot 0.5^{(i + 1) / 2}$}\IComment{Compute the variances for each point}
            \EndFor{}
            \State\Call{Recurse}{$\vec x$, $0$, $N$, $1$, $nrc$}
            \State\Return{$\vec x$}
        \EndFunction{}
        \Function{Recurse}{$\vec x$, $t_0$, $t_2$, $t$, $nrc$}
            \State{$t_1 \gets (t_0 + t_2) / 2$}
            \State{$x_{t_1} \gets \frac{1}{2}\cdot(x_{t_0} + x_{t_2}) + \Delta_t \cdot \mathrm{rand()}$}
            \If{$t < nrc$}\IComment{Recursively fill the rest of the vector $\vec x$}
                \State\Call{Recurse}{$\vec x$, $t_0$, $t_1$, $t+1$, $nrc$}
                \State\Call{Recurse}{$\vec x$, $t_1$, $t_2$, $t+1$, $nrc$}
            \EndIf{}
        \EndFunction{}
    \end{algorithmic}
    % \end{noindent}
    \caption{The random midpoint displacement algorithm in one dimension}\label{prob3:alg:rmd1d}
\end{algorithm}

This algorithm effectively generates a random heightmap above a regularly spaced segment of the real number line.
The extension of the algorithm to two dimensions would generate a heighmap above a square 2D grid --- effectively generating a landscape heightmap.

To do so, we must modify Algorithm~\ref{prob3:alg:rmd1d} to generate a square matrix of random values, and recursively quadsect this matrix until we have reached the recursion limit $nrc$.
This is shown in Algorithm~\ref{prob3:alg:rmd2d}.

\begin{algorithm}
    % \begin{noindent}
    \begin{algorithmic}
        \Function{RandomMidpoint2D}{$nrc$, $\sigma$, $seed$}
            \State{Seed the RNG with $seed$}
            \State{$N \gets 2^{nrc}$}\IComment{$N+1$ is the length of one side of the square matrix}
            \State{$X_{0,0}, X_{0,N}, X_{N,0}, X_{N,N} \gets \sigma \cdot \mathrm{rand()}$}\IComment{Initialize each of the matrix corners}
            \For{$i \in \{1, \dots, nrc\}$}
                \State{$\Delta_i \gets \sigma \cdot \frac{1}{2}^{(i + 1) / 2}$}
            \EndFor{}
            \State\Call{Recurse2D}{$X$, $(0, 0)$, $(N, 0)$, $(N, N)$, $(0, N)$, $t$, $nrc$}\IComment{Corners in CW order}
        \EndFunction{}
        \Function{Recurse2D}{$X$, $(x_0, y_0)$, $(x_2, y_0)$, $(x_2, y_2)$, $(x_0, y_2)$, $t$, $nrc$}
            \State{$x_1 \gets (x_0 + x_2) / 2$}
            \State{$y_1 \gets (y_0 + y_2) / 2$}
            \Statex\todoinline[caption=Question for McGough]{
                Ask McGough about this. Perturbing only the center adds an edge case, but eliminates repeated perturbation of the same indices. You don't want to perturb the same location four times\dots

                I guess we could try both and see what kind of results we get.
            }
            \State{$X_{x_0, y_1} \gets \frac{1}{2}\big(X_{x_0, y_0} + X_{x_0, y_2}\big) + \Delta_t \cdot \mathrm{rand()}$}\IComment{Potential race condition if parallelized.}
            \State{$X_{x_1, y_0} \gets \frac{1}{2}\big(X_{x_0, y_0} + X_{x_2, y_0}\big) + \Delta_t \cdot \mathrm{rand()}$}
            \State{$X_{x_2, y_1} \gets \frac{1}{2}\big(X_{x_2, y_0} + X_{x_2, y_2}\big) + \Delta_t \cdot \mathrm{rand()}$}
            \State{$X_{x_1, y_2} \gets \frac{1}{2}\big(X_{x_0, y_2} + X_{x_2, y_2}\big) + \Delta_t \cdot \mathrm{rand()}$}
            \State{$X_{x_1, y_1} \gets \frac{1}{4}\big(X_{x_0, y_1} + X_{x_1, y_0} + X_{x_2, y_1} + X_{x_1, y_2}\big) + \Delta_t \cdot \mathrm{rand()}$}\IComment{Perturb the center of the square}
            \If{ $t < nrc$}
                \State\Call{Recurse2D}{$X$, $(x_0, y_0)$, $(x_1, y_0)$, $(x_1, y_1)$, $(x_0, y_1)$, $t+1$, $nrc$}
                \State\Call{Recurse2D}{$X$, $(x_1, y_0)$, $(x_2, y_0)$, $(x_2, y_1)$, $(x_1, y_1)$, $t+1$, $nrc$}
                \State\Call{Recurse2D}{$X$, $(x_1, y_1)$, $(x_2, y_1)$, $(x_2, y_2)$, $(x_1, y_2)$, $t+1$, $nrc$}
                \State\Call{Recurse2D}{$X$, $(x_0, y_1)$, $(x_1, y_1)$, $(x_1, y_2)$, $(x_0, y_2)$, $t+1$, $nrc$}
            \EndIf{}
        \EndFunction{}
    \end{algorithmic}
    % \end{noindent}
    \caption{The 2D extension of Algorithm~\ref{prob3:alg:rmd1d}}\label{prob3:alg:rmd2d}
\end{algorithm}

Modifying both algorithms to work with fractional Brownian motion involves changing the $\Delta_t$ update step for each recursive level from
\begin{equation}
    \Delta_t = \sigma \cdot \frac{1}{2}^{(t + 1) / 2}\label{prob3:eqn:rmd-Delta}
\end{equation}
to
\begin{equation}
    \Delta_t = \sqrt{\frac{\sigma^2}{2^{2tH}} \big(1 - 2 ^ {2 H - 2}\big)}\label{prob3:eqn:fBm-Delta}
\end{equation}

Then the parameter $H$, called the Hurst exponent, describes the roughness of the fractal.
The book indicates lower values of $H$ should \textit{increase} the fractal roughness, while higher values will decrease the roughness.

Also note that the fractal's dimension can be recovered from $H$
\begin{equation}
    d = 2 - H\label{prob3:eqn:fractal-dimension-1d}
\end{equation}
for the 1D random midpoint algorithm, and
\begin{equation}
    d = 3 - H\label{prob3:eqn:fractal-dimension-2d}
\end{equation}
for the 2D version.
Thus smaller $H$ values \textit{increase} the fractal dimension, while larger values decrease the fractal dimension.

\subsection{Implementation}
\subsection{Results}
\subsection{Conclusion}
